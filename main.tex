\documentclass[a4paper, 11pt, uplatex]{classes/oecu-thesis}

\usepackage{subfiles}
\usepackage{listings}
\usepackage{plistings}
\usepackage{color}
\usepackage[dvipdfmx]{graphicx}
\usepackage{url}
\usepackage{siunitx}
\usepackage{enumerate}
\usepackage{paralist}
\usepackage{amsmath,amssymb}
\usepackage{mathtools}
\usepackage{times}
% \usepackage[newfloat]{minted}
\usepackage[hang,small,bf,labelsep=space]{caption}
\usepackage[subrefformat=parens]{subcaption}
\usepackage[dvipdfmx]{hyperref}
\usepackage{pxjahyper}
\usepackage{framed}
\usepackage[dvipdfmx]{pdfpages}
\usepackage[backend=biber,style=ieee]{biblatex}
\addbibresource{references.bib}
\captionsetup{compatibility=false}
\renewcommand{\lstlistingname}{プログラム}
\makeatletter
    \AtBeginDocument{
    \renewcommand*{\thelstlisting}{\arabic{lstlisting}}
    \@addtoreset{lstlisting}{section}}
\makeatother

\title{OCRを用いたレシート情報の取り込み機能を備えた家計簿アプリケーションの開発}
\author{松本 朝}
\date{{令和}\rensuji{7}年\rensuji{8}月}
\学生番号{HT22A103}
\指導教員{久松 潤之 准教授}

% 特別研究の場合はコメントをはずす。卒業研究の場合はコメントアウトする。
%\論文種別{特別研究論文}
\年度{令和7}

\所属{総合情報学部 情報学科}


\begin{document}
\maketitle
\pagenumbering{roman}
\subfile{chapters/abstract}

\tableofcontents
% 以下の二つは、学科が配布している論文例では示されていないが、あった方がよいのでつけている。
% このままつけたままでよい。ただし、ページ数としてはこの2ページはカウントされない。
\listoffigures
\listoftables

\cleardoublepage
\setcounter{page}{1}
\pagenumbering{arabic}

\subfile{chapters/intro}
\subfile{chapters/related}
\subfile{chapters/overview}
\subfile{chapters/environment}
\subfile{chapters/motion}
% \subfile{chapters/evaluation} % いったん、コメントアウト
\subfile{chapters/conclusion}
\subfile{chapters/acknowledgements}
\printbibliography[title=参考文献]

% cs コースの学生は以下のコメントをはずす & chapters/programs-list.tex を修正
% \appendix
% \subfile{chapters/programs-list}

\end{document}