%! TEX root = ../main.tex
\documentclass[main]{subfiles}

\begin{document}
\chapter{関連研究}
\label{cha:related}

% 具体的には、OCR技術および既存の家計簿アプリケーションについて説明する。
% 以下に、各研究および技術の特徴を示す。

% \section{OCR技術の関連研究}
% OCR について書く必要がない。。。

% Tesseract OCR\cite{Tesseract}は、Googleが開発したオープンソースのOCRエンジンであり、高精度な文字認識が可能である。
% TesseractはLSTM(Long Short-Term Memory)ベースのディープラーニングモデルを採用しており、手書き文字や多言語対応に優れている。
% 本システムでは、このTesseract OCRを利用して日本語レシートの文字認識を行っている。
% 特に、日本語モデル(jpn)を利用することで、複雑な漢字やカタカナ、数字の混在するレシートデータに対応している。

% 他のOCR技術としてMicrosoft Azure Computer Vision が挙げられる。
% Microsoft Azure Computer Vision は、Microsoftが提供するクラウドベースのOCRサービスであり、
% 文書レイアウト認識や手書きテキストの抽出に強みを持つ。\cite{AzureOCR} によると、このサービスは日本語を含む多言語に対応しており、
% 大量のデータ処理にも適している。特に、手書き文字や表形式の文書に対して高度な解析を行うことが可能である。
% しかし、Tesseract OCRと比較すると、クラウド環境への依存が大きく、プライバシー保護やコスト管理が課題となることが多い。

% Google Cloud Vision API は、Googleが提供するOCRサービスであり、テキスト認識だけでなく、
% 画像内のラベル付けや顔認識、オブジェクト検出など、多様な機能を備えている。
% \cite{GoogleVision} によれば、このAPIは高い認識精度を持ち、特に画像品質が悪い場合でも強力な文字認識を実現している。
% 本システムのようなローカル環境での処理とは異なり、クラウド環境を利用するため、膨大な計算リソースを活用できる利点がある。
% 一方で、インターネット接続が必須であり、継続的な使用にコストがかかるという課題もある。

% さらに、オープンソースのOCRライブラリとして EasyOCR が挙げられる。
% EasyOCRは、PyTorchをベースとしたOCRライブラリであり、多言語に対応し、比較的簡単に導入・利用できる点が特徴である。
% \cite{EasyOCR} によれば、特に手書き文字や複雑なフォントの認識において、Tesseractに匹敵する性能を持つとされている。
% ただし、特定の言語モデルやカスタムデータセットを使用する場合には、追加のトレーニングや設定が必要になることがある。

% 本システムでは、これらのOCR技術の中で、オフライン動作が可能で高いカスタマイズ性を持つ Tesseract OCR を採用している。
% 他のOCR技術と比較して、導入コストが低く、ローカル環境での使用に最適である点が選定理由となった。
% 一方で、今後の改良点として、Google Cloud Vision APIやAzure Computer Visionのようなクラウドベースのサービスを統合することで、
% 大量データ処理時の精度向上やレイアウト解析能力の強化を検討する余地がある。

% \section{既存の家計簿アプリケーションの関連研究}

本章では、既存の家計簿アプリケーションについて説明する。
既存の家計簿アプリケーションの関連研究として、Money Forward \cite{}とLINE家計簿\cite{} が挙げられる。
% 2つしかあげないのはすくないと思う。
% \cite{} は、まず、references.bib にアプリケーションの情報をかいてください(既に書いてるgithub が参考になるはずです)
Money Forward\cite{MoneyForwardApp}は、銀行口座やクレジットカード、電子マネーとの連携に優れた家計簿アプリケーションであり、
オンライン取引データを自動的に収集し、効率的に家計簿を作成することが可能である。
しかし、レシート撮影機能には重点が置かれておらず、OCR機能の対応範囲も限定的である。
また、データの可視化や分析機能には優れている一方で、レシートを基にした詳細な支出管理には対応していない点が課題である。

LINE家計簿\cite{LINEKakeibo}は、LINEアプリ内で利用可能という利便性を特徴としており、手軽に家計管理ができる。
しかし、レシート撮影機能を持たず、主に手動入力やクレジットカードのデータ同期を利用する設計となっている。
その結果、レシートを基にした支出管理や詳細なデータ分類には対応しておらず、
さらに支出データの視覚化や分析機能が簡易的である点が課題として指摘されている。

Moneytree\cite{MoneytreeApp}は、銀行口座、クレジットカード、電子マネーなどの情報を一元管理する機能に優れたアプリケーションである。
このアプリは、金融機関からの取引データを自動的に同期し、最新の支出状況を把握することが可能である。
しかし、レシート撮影機能を備えておらず、紙のレシートを基にした支出管理には対応していない。
また、現金取引の管理が手動入力に頼る点も課題として挙げられる。

家計簿 簡単お小遣い帳\cite{KakeiboSimpleApp}は、シンプルな操作性と軽量アプリとしての特長を持ち、手動入力を中心に利用する家計簿アプリケーションである。
このアプリは、直感的で分かりやすいインターフェースを提供しており、簡単な日常的支出の記録に適している。
一方で、レシート撮影機能や銀行口座との連携機能を持たず、デジタル化や自動化を求めるユーザーには適していない点が課題である。

表\ref{tab:featureComparisonKakeibo}に、既存のアプリケーションと本論文で開発するアプリケーションについてまとめる。
既存のアプリケーションであるMoney Forward、LINE家計簿、Moneytree、家計簿 簡単お小遣い帳は、それぞれ独自の強みを持っている。

Money ForwardとMoneytreeは銀行口座やクレジットカードとの連携機能に優れ、
オンライン取引データを自動収集して家計簿を効率的に作成できる点が特徴である。
一方、LINE家計簿はLINEアプリ内で簡単に利用できる手軽さが特徴であるが、レシートOCR機能や長期的な支出集計機能には対応していない。

家計簿 簡単お小遣い帳は、簡易的で手軽に利用できる家計簿アプリとして親しまれており、レシートOCR機能を備えている。
しかし、支出データの視覚化や長期的な支出の分析には不十分な点が課題として挙げられる。

これに対し、本論文で開発するアプリケーションは、カレンダー形式で支出データを表示する機能が特徴としてある。
この機能により、日付ごとに支出データを直感的に管理できるだけでなく、
OCR機能を活用してレシートから品目、金額、日付を自動抽出することで、データ登録の効率化を図っている。
また、手動入力やカテゴリ分け機能を設計し、ユーザーのニーズに応じたカスタマイズが可能となっている点が特徴である。

さらに、視覚化機能や長期的な支出集計機能も備えており、短期的な支出管理だけでなく、長期的な消費傾向の把握にも少し対応可能である。
このように、既存の家計簿アプリケーションが対応していないカレンダー形式の支出管理やOCRによる自動データ登録など、
独自の価値を提供するアプリケーションとして設計されている点が、本アプリケーションの大きな利点であることが確認できる。

\begin{table}[tbp]
    \centering
    \caption{各家計簿アプリケーションの特徴比較表}
    \label{tab:featureComparisonKakeibo}
    \small
    \begin{tabular}{|l|c|c|c|c|c|}
        \hline
        \textbf{機能} & \textbf{Money Forward} & \textbf{LINE家計簿} & \textbf{Moneytree} & \textbf{家計簿 簡単お小遣い帳} & \textbf{本アプリケーション} \\
        \hline
        銀行口座連携 & ○ & × & ○ & × & × \\
        \hline
        クレジットカード連携 & ○ & ○ & ○ & × & × \\
        \hline
        レシートOCR対応 & △(限定的) & × & ○ & ○ & ○ \\
        \hline
        支出データの視覚化 & ○ & △(簡素) & ○ & △ & ○ \\
        \hline
        カレンダー形式での表示 & × & × & × & × & ○ \\
        \hline
        長期的な支出集計 & ○ & × & ○ & △ & △ \\
        \hline
    \end{tabular}
\end{table}

\end{document}
%% 各種アプリケーションとのまるぺけ表をかいてください。
%% 西村とかの related.tex を参考にしてください。