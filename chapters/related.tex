%! TEX root = ../main.tex
\documentclass[main]{subfiles}

\begin{document}
\chapter{関連研究}
\label{cha:related}

本章では、関連研究として、本システムの開発において参照した主要な技術や類似システムについて概説する。
具体的には、OCR技術および既存の家計簿アプリケーションについて説明する。
以下に、各研究および技術の特徴を示す。

\section{OCR技術の関連研究}

Tesseract OCR\cite{Tesseract}は、Googleが開発したオープンソースのOCRエンジンであり、高精度な文字認識が可能である。
TesseractはLSTM(Long Short-Term Memory)ベースのディープラーニングモデルを採用しており、手書き文字や多言語対応に優れている。
本システムでは、このTesseract OCRを利用して日本語レシートの文字認識を行っている。
特に、日本語モデル(jpn)を利用することで、複雑な漢字やカタカナ、数字の混在するレシートデータに対応している。

他のOCR技術としてMicrosoft Azure Computer Vision が挙げられる。
Microsoft Azure Computer Vision は、Microsoftが提供するクラウドベースのOCRサービスであり、
文書レイアウト認識や手書きテキストの抽出に強みを持つ。\cite{AzureOCR} によると、このサービスは日本語を含む多言語に対応しており、
大量のデータ処理にも適している。特に、手書き文字や表形式の文書に対して高度な解析を行うことが可能である。
しかし、Tesseract OCRと比較すると、クラウド環境への依存が大きく、プライバシー保護やコスト管理が課題となることが多い。

Google Cloud Vision API は、Googleが提供するOCRサービスであり、テキスト認識だけでなく、
画像内のラベル付けや顔認識、オブジェクト検出など、多様な機能を備えている。
\cite{GoogleVision} によれば、このAPIは高い認識精度を持ち、特に画像品質が悪い場合でも強力な文字認識を実現している。
本システムのようなローカル環境での処理とは異なり、クラウド環境を利用するため、膨大な計算リソースを活用できる利点がある。
一方で、インターネット接続が必須であり、継続的な使用にコストがかかるという課題もある。

さらに、オープンソースのOCRライブラリとして EasyOCR が挙げられる。
EasyOCRは、PyTorchをベースとしたOCRライブラリであり、多言語に対応し、比較的簡単に導入・利用できる点が特徴である。
\cite{EasyOCR} によれば、特に手書き文字や複雑なフォントの認識において、Tesseractに匹敵する性能を持つとされている。
ただし、特定の言語モデルやカスタムデータセットを使用する場合には、追加のトレーニングや設定が必要になることがある。

本システムでは、これらのOCR技術の中で、オフライン動作が可能で高いカスタマイズ性を持つ Tesseract OCR を採用している。
他のOCR技術と比較して、導入コストが低く、ローカル環境での使用に最適である点が選定理由となった。
一方で、今後の改良点として、Google Cloud Vision APIやAzure Computer Visionのようなクラウドベースのサービスを統合することで、
大量データ処理時の精度向上やレイアウト解析能力の強化を検討する余地がある。

\section{既存の家計簿アプリケーションの関連研究}

また、既存の家計簿アプリケーションの関連研究として、Money ForwardとLINE家計簿が挙げられる。

Money Forwardは、銀行口座、クレジットカード、電子マネーなどの連携に優れた家計簿アプリケーションであり、
主にオンライン取引データの自動収集を基に家計簿を作成する機能を持つ。
しかし、レシート撮影機能には重点が置かれておらず、OCR機能の対応も限定的である。
また、データの可視化や分析には長けているものの、レシートを基にした詳細な支出管理には対応していない点が課題とされている\cite{MoneyForwardApp}。

LINE家計簿は、LINEアプリ内で家計簿を手軽に管理できることが特徴のアプリケーションである。
しかし、レシート撮影機能を持たず、主に手動入力やクレジットカードのデータ同期を利用する設計となっている。
そのため、レシートを基にした支出管理には対応しておらず、支出データの視覚化や分析機能も簡素であるため、
詳細な家計分析には向いていない点が課題である\cite{LINEKakeibo}。

これらのアプリケーションに対し、本アプリケーションはOCR機能を活用し、レシート画像から品目、金額、日付を自動的に抽出することで、
紙媒体の情報を効率的にデジタル化し、詳細かつ正確な支出管理を可能にしている。
さらに、手動入力やデータ編集機能を統合することで、OCRの認識エラーを補正し、
ユーザーの個別ニーズに応じたデータ管理を実現している点が特徴である。