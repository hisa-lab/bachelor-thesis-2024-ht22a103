%! TEX root = ../main.tex
\documentclass[main]{subfiles}

\begin{document}
\chapter{開発する Web アプリケーション}
\label{cha:overview}
\section{アプリケーションの構成}

開発したアプリケーションのシステム構成は、図\ref{fig:system}に示すとおりである。
本アプリケーションは、フロントエンドにReact、サーバーサイドにNode.jsを使用して構築される。
フロントエンドでは、ユーザーがレシート画像をアップロードし、解析結果を閲覧できるインターフェースを提供している。
Reactのコンポーネントベースの設計により、ユーザーインターフェースをモジュール化し、メンテナンス性を向上させた。
一方、サーバーサイドにはNode.jsを採用し、OCR解析やデータ処理のロジックを実装した。Node.jsの非同期処理能力を活用することで、
画像解析やデータ処理を行った。また、サーバーサイドでは、抽出されたデータを整形し、
必要に応じて外部システムやクライアント側に送信する役割を担っている。

\begin{figure}[tb]
    \begin{center}
        \fbox{
            \includegraphics[width=.8\linewidth]%
            {../figures/system.png}
        }
        \caption{アプリケーションの構成図}
        \label{fig:system}
    \end{center}
\end{figure}

\section{アプリケーションの機能}

本研究で開発したアプリケーションに実装された機能について述べる。
% 開発したアプリケーションの画面の全体図を図\ref{fig:overall}に示す。
本アプリケーションは、支出データの追加、カレンダー、手動での分類、合計金額を表示する機能で構成される。
各機能の役割について、以下で解説する。

% \begin{figure}[tb]
%     \begin{center}
%         \fbox{
%             \includegraphics[width=.8\linewidth]%
%             {../figures/overall.png}
%         }
%         \caption{アプリケーションの画面}
%         \label{fig:overall}
%     \end{center}
% \end{figure}

\subsection{画像アップロード機能}

本研究では、OCR処理対象として、事前に選定した3店舗(ダイエーグルメシティ、ファミリーマート、ココス)のレシートを使用している。
以下にその3店舗のレシート図 \ref{fig:daie}--\ref{fig:cocos}に示す。
\begin{figure}[tb]
    \begin{center}
        \fbox{
            \includegraphics[width=.7\linewidth]%
            {../figures/daie.jpeg}
        }
        \caption{ダイエーグルメシティのレシート}
        \label{fig:daie}
    \end{center}
\end{figure}
\begin{figure}[tb]
    \begin{center}
        \fbox{
            \includegraphics[width=.8\linewidth]%
            {../figures/familymart.jpeg}
        }
        \caption{ファミリーマートのレシート}
        \label{fig:familymart}
    \end{center}
\end{figure}
\begin{figure}[tb]
    \begin{center}
        \fbox{
            \includegraphics[width=0.6\linewidth, height=0.9\textheight]%
            {../figures/cocos.jpeg}
        }
        \caption{ココスのレシート}
        \label{fig:cocos}
    \end{center}
\end{figure}
このため、他店舗のレシートに関しては読み取ることが出来ない。今後の改良により、対象店舗の拡張も視野に入れている。
本システムにおける画像アップロード機能は、ユーザーがレシートデータを画像形式で提供できるようにする主要なモジュールである。
この機能はOCR処理の起点となり、レシート画像から日付、品目、金額などの情報を抽出する役割を果たす。

まず、ユーザーがローカルデバイス内の画像ファイルを選択し、インターフェース上でアップロードする。
このアップロードされた画像は、Tesseract.jsを用いたOCRエンジンに渡され、文字認識処理が実行される。
OCRエンジンでは、画像内の文字列情報がテキスト形式に変換され、レシート内の項目ごとに必要な情報(品目や金額)が抽出される。

また、OCR処理で日付が正確に認識できない場合や、情報が不足している場合には、エラーが表示され、
ユーザーは手動で日付を入力することでデータを補足できる。
レシートや明細書が発行されない取引(例:自動販売機での購入)についても手動でデータを入力することができる。
この設計により、OCR精度の限界による不完全なデータ入力を補完している。
\subsection{カレンダー機能}

本システムにおけるカレンダー機能は、支出データを日付ごとに確認できるようにするためのモジュールである。
具体的には「何年何月何日」の形式で管理している。
この機能は、ユーザーが指定した特定の日付の支出データを素早く取得し、必要な情報にアクセスできる操作性を提供する。

まず、画面に配置されたカレンダーを利用し、ユーザーが対象の日付を選択する。
この操作により、システムは選択された日付を基に、登録済みの支出データから該当する日付のデータのみを抽出する。
抽出されたデータは、品目、金額、カテゴリとともに画面上に一覧として表示される。

本機能の実装には、react-calendarライブラリを使用している。
選択された日付は、JavaScriptのDateオブジェクトとして取得され、システム内部でフォーマット処理を行い
、登録データの日付フォーマットと一致させて検索処理を行っている。
この処理により、ユーザーがカレンダー上で日付を選択するだけで、その日に該当する支出データが即座に表示される仕組みが実現されている。
\subsection{手動での分類機能}

% カテゴリの欄の画像を\ref{fig:category}以下に示す。
% \begin{figure}[tb]
%     \begin{center}
%         \fbox{
%             \includegraphics[width=.8\linewidth]%
%             {../figures/category.png}
%         }
%         \caption{カテゴリ入力欄}
%         \label{fig:category}
%     \end{center}
% \end{figure}
本システムにおける手動での分類機能は、OCRや手動入力によって登録された支出データに対し、
ユーザーが各項目に適切なカテゴリを付与するための機能である。
この機能は、ユーザーが支出を目的別または種類別に分類し、データの管理性や視認性を向上させることを目的としている。
特に、複数のカテゴリにまたがるレシートに対応できる点が、家計管理を実現する上で重要である。

具体的には、支出データの一覧表示画面において、各項目に対して直接入力可能なカテゴリ欄を設けている。
ユーザーは、この入力欄を利用して自由にカテゴリ名を入力でき、例えば「食品」「交通費」「日用品」といった分類を追加できるようにしている。
また、分類の際に選択肢を制限することなく自由入力を可能にしている。

さらに、分類された支出データを基に、カテゴリごとの合計金額を計算し、一覧として提示する機能も実装している。
この情報は、ユーザーがカテゴリごとの支出を把握し、家計管理や予算計画の策定に活用するための基礎データとして機能する。
これにより、ユーザーは自身の消費傾向を把握し、支出の見直しや節約意識の向上といった行動変容につなげることが可能となる。
\subsection{合計金額を表示する機能}

本システムにおける合計金額を表示する機能は、登録された支出データの総額を自動的に算出し、ユーザーに提示する機能である。
この機能は、全体の支出状況を一目で把握できるように設計されており、家計管理や予算計画を支援する役割を果たしている。
加えて、支出の傾向を視覚的に把握することで、ユーザーが日常の消費行動を見直すきっかけにもなる。

具体的には、OCRや手動入力で登録された個々の支出データの金額フィールドを基に、合計金額をリアルタイムで計算し、画面上部に表示する仕組みとなっている。
また、合計金額は、カテゴリごとの分類に基づいたサブ合計や、特定の日付に限定した支出合計といった形式でも表示される。
これにより、ユーザーは全体の支出だけでなく、カテゴリ別や日付別の詳細な支出状況も確認することが可能である。

この機能の実装においては、Reactの状態管理機能を活用している。
支出データが追加・削除・編集された際に、リアルタイムで合計金額が再計算される仕組みを採用することで、
常に最新の情報が正確に画面上に反映されるようになっている。
このような構成により、ユーザーは支出の過不足を即座に把握し、無駄な出費を抑えるための判断材料として活用することができる。

\end{document}