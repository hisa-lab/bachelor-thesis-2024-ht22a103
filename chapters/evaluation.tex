%! TEX root = ../main.tex
\documentclass[main]{subfiles}

\begin{document}
\chapter{アプリケーションの評価}
\label{cha:evaluation}
この章では、開発したアプリケーションの有用性や改善点を明らかにするために行った評価について述べる。
本評価は、実際のユーザーを想定した被験者4名に対し、画面共有によってアプリケーションを操作している場面を視聴してもらい、操作性やOCR精度、
将来的な利用意向などの観点から主観的評価を行ったものである。

\subsection{評価方法}
被験者には、私が実際にレシート画像をアップロードし、OCR処理、編集、分類、合計金額の確認といった一連の操作を行い、それを視聴してもらった。
この際、本アプリケーションではOCR処理対象として、事前に選定した3店舗のレシートしか対応していないことも説明している。
そして、表\ref{tab:evaluation_criteria}に示す評価項目に基づいてアンケートに回答してもらった。
評価は5段階評価(1:非常に悪い、5:非常に良い)および自由記述形式で実施した。

\begin{table}[htbp]
\centering
\caption{アプリケーションの評価項目}
\label{tab:evaluation_criteria}
\begin{tabular}{lll}
\hline
評価項目 & 説明 & 評価形式 \\
\hline
使いやすさ & 画面構成や操作が直感的であるか & 5段階評価 \\
OCRの正確さ & 品目・金額・日付が正確に抽出できたか & 5段階評価 \\
今後も使いたいか & 継続的に利用したいと感じたか & 5段階評価 \\
改善してほしい点 & 操作や機能における不便な点 & 自由記述 \\
追加してほしい機能 & 今後搭載してほしいと感じた機能 & 自由記述 \\
\hline
\end{tabular}
\end{table}

\section{アプリケーションの評価}
5段階評価の結果を以下の表\ref{tab:evaluation_results}に示す。

\begin{table}[htbp]
\centering
\caption{評価結果(5段階評価、4名分)}
\label{tab:evaluation_results}
\begin{tabular}{cccc}
\hline
被験者 & 使いやすさ & OCRの正確さ & 今後も使いたいか \\
\hline
A & 2 & 3 & 2 \\
B & 1 & 3 & 2 \\
C & 2 & 3 & 2 \\
D & 1 & 2 & 2 \\
\hline
平均 & 1.5 & 2.75 & 2 \\
\hline
\end{tabular}
\end{table}

使いやすさの項目では、評価は1または2に集中し、平均値は1.5であった。
OCRの正確さに関しては、すべての被験者が3または2を選択しており、平均値は2.75との結果となった。
今後も使いたいかという項目では、全員が2を選択し、平均値は2となった。
全体として、使いやすさと継続利用意向については低めの評価が得られた一方、OCRの精度に対しては良くもないが悪くもない意見にまとまった。

次に自由記述の結果を以下の表\ref{tab:improvement_requests}に示す。
\begin{table}[htbp]
\centering
\caption{改善してほしい点(自由記述)}
\label{tab:improvement_requests}
\begin{tabular}{ll}
\hline
評価者 & 内容 \\
\hline
A & 対応するレシートの種類が少ない \\
B & 一部のレシートが正しく認識されない \\
C & アプリの構成が分かりづらい \\
D & 情報が画面に詰まっており見づらい \\
\hline
\end{tabular}
\end{table}


\begin{table}[htbp]
\centering
\caption{追加してほしい機能(自由記述)}
\label{tab:additional_features}
\begin{tabular}{ll}
\hline
評価者 & 内容 \\
\hline
A & 一度に複数枚のレシートを登録する機能 \\
B & 支出額を週・月・年単位で集計できる機能 \\
C & アプリ内で直接写真を撮影できる機能 \\
D & 通知機能(支出入力のリマインダーなど) \\
\hline
\end{tabular}
\end{table}

自由記述式の質問では、各被験者から具体的な改善点が寄せられた。
評価者Aは「対応するレシートの種類が少ない」と回答し、対応フォーマットの限定性に対する指摘が見られた。
評価者Bからは「一部のレシートが正しく認識されない」というOCR処理の不完全性に関する意見が挙げられた。
評価者Cは「アプリの構成が分かりづらい」と述べており、UIや操作フローに対する不満が示されている。
評価者Dは「情報が画面に詰まっており見づらい」と記述しており、画面レイアウトや視認性への改善要望が見られた。

さらに、「追加してほしい機能」に関する自由記述では、ユーザーの多様な利用ニーズが反映された内容が得られた。
評価者Aは「一度に複数枚のレシートを登録する機能」を挙げており、入力作業の効率化を望んでいるのがうかがえる。
評価者Bは「支出額を週・月・年単位で集計できる機能」の追加を求めており、長期的な支出分析や家計管理の高度化を求める傾向が示された。
また、評価者Cは「アプリ内で直接写真を撮影できる機能」を希望しており、入力手順の簡略化と即時性の向上が望まれていることが分かる。
評価者Dからは「通知機能(支出入力のリマインダーなど)」が求められており、継続的な利用を促すための補助的機能の必要性が示唆された。

\subsection{評価結果の考察}

評価結果を総合的に考察すると、本アプリケーションには一定の実用性が認められるものの、
ユーザー体験の観点からは改善の余地が大きいことが明らかとなった。
特に「使いやすさ」の評価が平均1.5と低く、ユーザーインターフェースや操作性に対する不満が大きいことが示された。
自由記述においても、「アプリの構成が分かりづらい」「情報が画面に詰まっており見づらい」といった具体的な意見が複数挙げられており、
視認性や導線設計の見直しが求められる。

一方で、「OCRの正確さ」に関する評価は平均2.75と比較的高く、一定の認識精度が評価されたと考えられる。
ただし、評価者Bによる「一部のレシートが正しく認識されない」という意見に代表されるように、
全てのフォーマットに対応できていない現状が課題として残る。
対応可能なレシートの種類を拡充することで、さらなるユーザビリティの向上が見込まれる。

「今後も使いたいか」という継続利用意向の平均が2にとどまった点は、本アプリケーションの現段階での実用性の限界を示唆している。
自由記述からは、ユーザーがアプリに求める機能として、「複数レシートの一括登録」「集計機能」「カメラ撮影機能」「通知機能」などが挙げられており、
これらの機能拡張が実現されれば、継続的な利用意欲の向上が期待される。
総じて、現行バージョンは基本的な機能を実装できている一方で、UIの改善、対応フォーマットの拡充、利便性を高める追加機能の実装などが今後の課題である。
これらのフィードバックを反映し、アプリケーションをより実用的で魅力的なものへと発展させていくことが望まれる。

\end{document}
