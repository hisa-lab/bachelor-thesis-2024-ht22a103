%! TEX root = ../main.tex
\documentclass[main]{subfiles}

\begin{document}
\chapter{アプリケーションの評価}
\label{cha:evalidation}

この章では、開発したアプリケーションの有用性や改善点を明らかにするために行った評価について述べる。
評価は、アプリケーションの動作画面を視聴してもらい、その使用感や改善点についてアンケートを通じて意見を収集する形で実施した。
その結果を表\ref{tab:survey_results}に示す。

\begin{table}[tb]
    \caption{アンケート結果のまとめ}
    \label{tab:survey_results}
    \centering
    \begin{tabular}{|c|c|p{5cm}|p{5cm}|}
        \hline
        {使いやすさ} & {今後も使いたいか} & {他に欲しいと思う機能} & {感想} \\
                \hline
                2 & 2 & 対応するレシートの種類を増やしてほしいのと、一度に複数のレシートを登録できる機能があると便利だと感じました。 & 直感的で使いやすい \\
                \hline
                3 & 2 & 支出額を1週間、1ヶ月、年間単位で把握できる機能が欲しい & 一部のレシートが対応していなくて、不便に感じた \\
                \hline
                2 & 1 & 画面デザインをもっと見やすく改善してほしいのと、支出にメモを追加できる機能があると助るかも &画面の情報が固まっててみにくい\\
                \hline
        \end{tabular}
    \end{table}        

アンケートでは、「使いやすさ」「今後も使いたいか」について5段階評価で回答を求め、
自由記述で「他に欲しいと思う機能」や「感想」を記入してもらった。
その結果、使いやすさについては評価が2や3に集中し、全体的に期待を超える評価には至らなかった。
また、「今後も使いたいか」の評価も1や2に偏り、継続的な利用の動機付けには課題があることが分かった。

自由記述の回答では、対応するレシートの種類を増やすことや、
一括登録機能の追加が求められていることが分かった。
また、1週間、1ヶ月、年間単位での支出集計機能の要望が挙げられ、長期的な支出管理への対応が必要であることが示唆された。
さらに、画面デザインの改善や、支出にメモを追加する機能の要望もあり、情報の視認性や管理性を向上させる工夫が求められている。

評価全体を通じて、アプリケーションの基本的な設計は直感的で使いやすいと一定の評価を受けたものの、
対応レシートの範囲や機能拡張、画面デザインなど、改善すべき課題が浮き彫りになった。
これらのフィードバックを基に、今後の改良を進めていく必要がある。

\end{document}