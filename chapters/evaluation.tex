%! TEX root = ../main.tex
\documentclass[main]{subfiles}

\begin{document}
\chapter{アプリケーションの評価}
\label{cha:evalidation}
この章では、開発したアプリケーションの有用性や改善点を明らかにするために行った評価について述べる。
具体的には、数人のユーザーに実際にアプリケーションを使用してもらい、アンケートを通じてその使用感や改善点についての意見を収集した。
これらのフィードバックを基に、アプリケーションの機能性、使いやすさ、さらには今後の改良可能性について分析を行う。

以下に、評価方法、アンケート内容、結果の分析、そして得られた知見を示す。
本評価では、開発したアプリケーションを実際に利用してもらい、その使用感や改善点についてアンケートを通じて意見を収集する形式を採用した。
アンケート内容は、自由記述形式でアプリケーションに対する感想や、欲しいと感じる機能について記入してもらい、
さらに5段階評価(1が最低、5が最高)で「今後も使いたいか」と「使いやすさ」を回答してもらう形式とした。評価には4名のユーザーが参加した。
アンケート結果をまとめた表を下に示す\ref{tab:survey_results}。

\begin{table}[tb]
    \caption{アンケート結果のまとめ}
    \label{tab:survey_results}
    \centering
    \begin{tabular}{|c|c|p{3cm}|p{3cm}|}
        \hline
        使いやすさ & 今後も使いたいか & 他に欲しいと思う機能 & 感想 \\
        \hline
        4 & 4 & 支出一覧を省略できる機能 & 直感的で使いやすい \\
        \hline
        4 & 2 & メモ機能、1ヶ月単位での合計表示機能 & 家計簿アプリなので1日ではなく1ヶ月での合計が見られると良い \\
        \hline
        4 & 3 & 支出額を1週間、1ヶ月、年間単位で把握できる機能、一括読み込み機能 & 買った商品と値段が見やすくて良い。一括で読み込めるとさらに便利 \\
        \hline
        2 & 2 & 写真撮影機能、一定以上の金額で通知する機能 & シンプルな画面だが情報が詰まりすぎて見にくい \\
        \hline
    \end{tabular}
\end{table}

使いやすさの評価では、5段階評価で3人が4と答える結果が得られ、全体的に良好な評価を受けた。
この結果から、アプリケーションのインターフェースが直感的であり、基本的な操作がユーザーにとって理解しやすいことが示唆される。
一方で、一部のユーザーから「情報が固まっていて見づらい」といった指摘があり、
画面の情報配置や視覚的なデザインにおいて改善の余地があることが明らかとなった。

今後も使いたいかの評価では、5段階評価で2人が4、1人が3、2人が2という結果が得られた。
評価が分散する結果となり、基本的な機能に満足しているユーザーがいる一方で、継続利用に対する動機付けが十分ではないユーザーもいることが示された。
特に評価が低かった理由として、他に欲しいと思う機能で挙げられていたが「1ヶ月単位や年間単位での支出の集計が見られない」
「一括で複数のレシートを処理する機能が欲しい」といった具体的な改善要望が挙げられている。

また、他に欲しいと思う機能については、ユーザーから多様な要望が寄せられた。
レシートの枚数が増えることで支出一覧が大きくなり、全体が見づらくなるという課題が指摘され、支出一覧を省略表示できる機能が求められている。
また、支出データにコメントを追加できるメモ機能が欲しいという意見もあり、
これにより購入理由や背景を記録することで、数値データ以上の価値を提供できるようになると考えられる。

さらに、今後も使いたいかの評価でも触れたように、1週間、1ヶ月、年間単位で支出額を把握できる長期的な集計機能が求められている。
この機能は、短期的な支出管理だけでなく、長期的な消費傾向を把握するための有用な手段として期待されており、
予算の計画や見直しに役立つものとなる。また、一括で複数のレシートを読み込む機能についても要望が寄せられている。
この機能を導入することで、レシートを溜め込んでしまうユーザーでも効率的にデータを登録でき、
日常的な家計管理の負担を軽減することができると考えられる。

また、一定以上の金額を使った際に通知を送る機能についても要望があった。
この機能は予算管理を補助し、支出を抑制するための手助けとなる可能性がある。
さらに、アプリ内で直接写真を撮影し、レシートを登録できる機能があると便利だという意見も挙げられた。

以上の結果から、本アプリケーションは基本的な使いやすさや直感的な操作性において一定の評価を得ていることが示された。
一方で、継続的な利用においては、さらなる機能の追加や利便性の向上が求められている。
特に、長期的な支出の傾向を把握できる集計機能や、複数のレシートを一括で処理する機能など、ユーザーの具体的なニーズに基づいた改善が必要である。
また、視覚的な情報整理や、メモ機能などの細かな機能追加も、より多様なユーザー層に対応するための鍵となる。

これらの課題に対応することで、アプリケーションの実用性と満足度をさらに向上させることができ、継続的な利用を促進する可能性が高い。
今後は、これらのフィードバックを基に機能の改善や追加を検討し、ユーザーの期待に応えるシステムの実現を目指していくことが重要である。

\end{document}