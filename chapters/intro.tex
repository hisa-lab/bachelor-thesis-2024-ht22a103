%! TEX root = ../main.tex
\documentclass[main]{subfiles}

\begin{document}

\chapter{はじめに}
\label{cha:intro}

家計管理のデジタル化が進む中、家計簿アプリは多くの利用者にとって身近なツールとなっている。
しかし、現在主流の家計簿アプリにはいくつかの課題が存在する。
その多くは手動で数字を入力する形式か、ATMや銀行口座、クレジットカードと連携してデータを管理する形式を採用している。
ATMや銀行との連携については、情報漏洩への懸念があり、利用を控える人も少なくない。
また、手動入力の方法は手間がかかり、利用継続の障壁となりやすい。

本論文では、ATMや銀行口座との連携に依存せず、レシートなどの情報を基に家計簿を管理できるアプリケーションを開発することで、
現行の家計簿アプリが抱える課題の解決を目指す。このアプリケーションでは、光学式文字認識(OCR)技術を活用する。
OCRは、画像内に含まれる文字情報をテキストデータに変換する技術であり、レシートの写真を撮ることでその内容をデータ化することが可能である。

ただし、OCRが提供するのは画像から文字への変換であり、取得したテキストデータをそのまま家計簿に利用することはできない。
本研究では、このデータを家計簿に適した形式に加工し、各品目や金額を分類可能な形に変換する仕組みを実装する。
これにより、OCR技術を家計簿管理の実用的なツールとして活用する。

また、レシートや明細書が発行されない取引(例:自動販売機での購入)については、手動入力機能を併設することで、利用者が自身の取引を補完できる仕組みを提供する。
これらの取り組みにより、利用者はATMや銀行との連携に依存せず、柔軟かつ効率的に家計管理を行えるようになる。
本論文では、これらの技術的な工夫や設計について詳述し、提案するアプリケーションの有用性を示す。

% OCR でできるのは、写真→文字の変換でしょう。。。君も書いてるように。
% OCR 使ったからといって、写真→家計簿のデータにはならない。その分補わなければならないが、そのようなことが書かれていない。
% 日本語がおかしいです。「補完するため->記録可能であることが特徴」となっている。。。おかしい。

しかし、単にレシートの情報をそのまま記録するだけでは、家計管理における柔軟性が不十分である。
例えば、コンビニでの買い物において、「食費」として分類したい商品と「生活必需品」として分類したい商品が混在する場合がある。
本論文では、この問題を解決するために、レシートから抽出した情報を整理する仕組みを導入する。
この方法により、利用者は商品ごとに適切な分類を簡単に行うことができる。

% 行をどうするって、実装ベースの詳細度が高い内容なので、イントロではかかないでください。
現行の家計簿アプリには、OCRを利用してレシートを取り込む機能を備えているものもある。
しかし、これらのアプリは品目の自動分類や編集の効率性に課題を抱えている場合が多い。
例えば、「Zaim」では写真からデータを取得することは可能であるが、
品目ごとの分類は手動で行う必要がある。% 手動で行えるのなら柔軟性十分あるよ。。。。
また、「マネーフォワード」では自動的に見やすいグラフや表を作成するものの、
数字の入力が必要であり、分類作業を効率化する機能は限定的である。

本論文では、OCR技術を用いてレシートから抽出した情報を整理し、自由にカテゴリ分けできる仕組みを開発することを目的とする。
さらに、利用者が設定した分類に基づき、比較したい項目を直感的に把握できるカスタマイズ可能なグラフ作成機能を提供する。
このシステムにより、ATMや銀行との連携に依存せず、かつ柔軟で効率的な家計管理を実現することを目指す。

本論文の構成は以下の通りである。まず \ref{cha:related} 章では、関連研究について述べる。さらに、
\ref{cha:exfig} 章では、図表の作成の仕方を述べる。
最後に、\ref{cha:conclusion} 章では、本論文のまとめと今後の課題について述べる。

\end{document}
