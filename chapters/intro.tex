
%! TEX root = ../main.tex
\documentclass[main]{subfiles}

\begin{document}

\chapter{はじめに}
\label{cha:intro}

家計管理のデジタル化が進む中、家計簿アプリは多くの利用者にとって身近なツールとなっている。
しかし、現在主流の家計簿アプリにはいくつかの課題が存在する。
その多くは手動で数字を入力する形式か、ATMや銀行口座、クレジットカードと連携してデータを管理する形式を採用している。
ATMや銀行との連携については、情報漏洩への懸念があり、利用を控える人も少なくない。
また、手動入力の方法は手間がかかり、利用継続の障壁となりやすい。

こうした課題を踏まえ、レシートの写真を撮るだけで簡単に家計簿を作成・管理できる仕組みが求められている。
本研究では、この実現に向けて光学式文字認識(OCR)技術を活用する。
OCRは画像内の文字情報をテキストデータに変換する技術であり、レシートを写真に撮るだけで家計簿のデータを自動的に取得することが可能となる。

しかし、現行の家計簿アプリには、OCRを利用してレシートを取り込む機能を備えているものの、データの分類や編集の柔軟性に欠けるものが多い。
例えば、「Zaim」では写真からデータを取得することは可能であるが、品目ごとの分類は手動で行う必要がある。
また、「マネーフォワード」では自動的に見やすいグラフや表を作成するものの、数字の入力が必要であり、分類作業を効率化する機能は限定的である。

本研究では、OCR技術を用いてレシートから抽出した情報を行単位で整理し、自由にカテゴリ分けできる仕組みを開発することを目的とする。
さらに、レシートがない取引(例:自動販売機での購入)に対応するため、手動入力機能も併設する。
また、利用者が設定した分類に基づき、比較したい項目を直感的に把握できるカスタマイズ可能なグラフ作成機能を提供する。
このシステムにより、家計管理の効率化と利便性向上を図るとともに、個々のニーズに応じた柔軟な管理を可能とする。

本論文の構成は以下の通りである。まず \ref{cha:related} 章では、関連研究について述べる。さらに、
\ref{cha:exfig} 章では、図表の作成の仕方を述べる。
最後に、\ref{cha:conclusion} 章では、本論文のまとめと今後の課題について述べる。

\end{document}