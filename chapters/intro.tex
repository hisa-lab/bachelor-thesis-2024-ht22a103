%! TEX root = ../main.tex
\documentclass[main]{subfiles}

\begin{document}

\chapter{はじめに}
\label{cha:intro}

家計管理のデジタル化が進む中、家計簿アプリは多くの利用者にとって身近なツールとなっている。
しかし、現在主流の家計簿アプリにはいくつかの課題が存在する。
その多くは手動で金額を入力する形式か、ATMや銀行口座、クレジットカードと連携してデータを管理する形式を採用している。
ATMや銀行との連携については、情報漏洩への懸念があり、利用を控える人も少なくない。
また、手動入力は手間がかかり、利用継続の障壁となりやすい。

本論文では、ATMや銀行口座との連携に依存せず、レシートなどの情報を基に家計簿を管理できるアプリケーションを開発することで、
現行の家計簿アプリが抱える課題の解決を目指す。このアプリケーションでは、光学式文字認識(OCR)技術を活用する。
また、レシートや明細書が発行されない取引(例:自動販売機での購入)については、手動入力機能を併設することで、
利用者が自身の取引を補完できる。

さらに、買い物において「食費」として分類したい商品と「生活必需品」として分類したい商品が同じレシートに含まれる場合がある。
このようなケースに対応するため、レシートから取得したデータを整理し、利用者が手動で編集および分類できる仕組みを提案する。
% 君のアプリケーションだと、この問題発生しないの???
% 君のアプリケーションでこの問題(分類を手作業しないといけない)を解決するのなら、そのことを、この後の段落に書いてください。
% いまは「自由にカテゴリ分けできる」と書いてます。ここをかきなおしたらいいです。

本論文では、この課題に対処するため、OCR技術を用いてレシートから情報を抽出し、そのデータを整理する仕組みを提案する。
また、抽出した情報を基に、品目を手動で分類できる機能を開発し、利用者が必要な情報を簡単に記録できる仕組みを構築する。

本論文の構成は以下の通りである。まず \ref{cha:related} 章では、関連研究について述べる。さらに、
\ref{cha:exfig} 章では、図表の作成の仕方を述べる。
最後に、\ref{cha:conclusion} 章では、本論文のまとめと今後の課題について述べる。

\end{document}
