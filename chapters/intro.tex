%! TEX root = ../main.tex
\documentclass[main]{subfiles}

\begin{document}

\chapter{はじめに}
\label{cha:intro}

家計管理のデジタル化が進む中、家計簿アプリは多くの利用者にとって身近なツールとなっている。
しかし、現在主流の家計簿アプリにはいくつかの課題が存在する。
その多くは手動で金額を入力する形式か、ATMや銀行口座、クレジットカードと連携してデータを管理する形式を採用している。
ATMや銀行との連携については、情報漏洩への懸念があり、利用を控える人も少なくない。
また、手動入力は手間がかかり、利用継続の障壁となりやすい。

本論文では、ATMや銀行口座との連携に依存せず、レシートなどの情報を基に家計簿を管理できるアプリケーションを開発することで、
現行の家計簿アプリケーションが抱える課題の解決を目指す。このアプリケーションでは、光学式文字認識(OCR)技術を活用する。
また、レシートや明細書が発行されない取引(例:自動販売機での購入)については、手動入力機能を併設することで、
利用者が自身の取引を補完できる。
さらに、買い物において「食費」として分類したい商品と「生活必需品」として分類したい商品が同じレシートに含まれる場合がある。
本論文で開発するアプリケーションは、レシートから取得したデータを整理し、利用者が手動で編集
および分類することで、このようなケースに対応する。

本論文の構成は以下の通りである。まず \ref{cha:related} 章では、関連研究について述べる。さらに、
\ref{cha:overview} 章では、開発するWebアプリケーションと機能について述べる。
次に\ref{cha:environment}章では開発環境について述べる。
その次に\ref{cha:motion} 章では、開発したアプリケーションの動作や使用した際の流れを記述する。
そして、\ref{cha:evalidation}章では、アプリケーションを数人に使用してもらいアンケートを取るその結果をもとに評価を行い、その内容を記述する
最後に、\ref{cha:conclusion} 章では、本論文のまとめと今後の課題について述べる。

\end{document}
