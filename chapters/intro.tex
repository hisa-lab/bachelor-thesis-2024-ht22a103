%! TEX root = ../main.tex
\documentclass[main]{subfiles}

\begin{document}

\chapter{はじめに}
\label{cha:intro}

家計管理のデジタル化が進む中、家計簿アプリは多くの利用者にとって身近なツールとなっている。
しかし、現在主流の家計簿アプリにはいくつかの課題が存在する。
その多くは手動で数字を入力する形式か、ATMや銀行口座、クレジットカードと連携してデータを管理する形式を採用している。
ATMや銀行との連携については、情報漏洩への懸念があり、利用を控える人も少なくない。
また、手動入力は手間がかかり、利用継続の障壁となりやすい。

本論文では、ATMや銀行口座との連携に依存せず、レシートなどの情報を基に家計簿を管理できるアプリケーションを開発することで、
現行の家計簿アプリが抱える課題の解決を目指す。このアプリケーションでは、光学式文字認識(OCR)技術を活用する。
また、レシートや明細書が発行されない取引(例:自動販売機での購入)については、手動入力機能を併設することで、利用者が自身の取引を補完できる仕組みを提供する。

しかし、単にレシートの情報をそのまま記録するだけでは、家計管理における柔軟性が不十分である。
例えば、コンビニでの買い物において、「食費」として分類したい商品と「生活必需品」として分類したい商品が混在する場合がある。
本論文では、この問題を解決するために、手入力で情報を入力する仕組みを導入する。
この方法により、利用者は商品ごとに適切な分類を簡単に行うことができる。
% 柔軟性の実現のために手入力とか、あまりにもポジティブに書きすぎです。。。単に、変更に対応してるくらいで書いてください。

現行の家計簿アプリには、OCRを利用してレシートを取り込む機能を備えているものもある。
しかし、これらのアプリは品目の自動分類や編集の効率性に課題を抱えている場合が多い。
例えば、「Zaim」では写真からデータを取得することは可能であるが、
品目ごとの分類は手動で行う必要がある。
また、「マネーフォワード」では自動的に見やすいグラフや表を作成するものの、
数字の入力が必要であり、分類作業を効率化する機能は限定的である。
% 数字の入力って、なんのことかわからないです。。。

本研究では、OCRを活用してレシートから文字データを効率的に取得し、利用者が手動で分類分けを行える仕組みを提供する。
この設計では、レシートを処理し、対応するカテゴリに手動で割り振るインターフェースを実装することで、操作の簡便性と柔軟性を確保する。

本論文では、OCR技術を用いてレシートから抽出した情報を整理し、自由にカテゴリ分けできる仕組みを開発する。
さらに、利用者が設定した分類に基づき、比較したい項目を直感的に把握できるカスタマイズ可能なグラフ作成機能を提供する。
このシステムにより、ATMや銀行との連携に依存せず、かつ柔軟で効率的な家計管理を実現することを目指す。

本論文の構成は以下の通りである。まず \ref{cha:related} 章では、関連研究について述べる。さらに、
\ref{cha:exfig} 章では、図表の作成の仕方を述べる。
最後に、\ref{cha:conclusion} 章では、本論文のまとめと今後の課題について述べる。

\end{document}
