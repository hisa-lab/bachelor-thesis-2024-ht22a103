%! TEX root = ../main.tex
\documentclass[main]{subfiles}

\begin{document}

\chapter{はじめに}
\label{cha:intro}

家計管理のデジタル化が進む中、家計簿アプリは多くの利用者にとって身近なツールとなっている。
しかし、現在主流の家計簿アプリにはいくつかの課題が存在する。
その多くは手動で数字を入力する形式か、ATMや銀行口座、クレジットカードと連携してデータを管理する形式を採用している。
ATMや銀行との連携については、情報漏洩への懸念があり、利用を控える人も少なくない。
また、手動入力の方法は手間がかかり、利用継続の障壁となりやすい。

こうした課題を踏まえ、ATMや銀行と連携せずに家計簿を効率的に管理できる仕組み仕組みが必要だと考えた。
本論文では、この課題に対する解決策として、光学式文字認識(OCR)技術を活用する。
OCRは画像内の文字情報をテキストデータに変換する技術であり、
レシートを写真に撮るだけで家計簿のデータを自動的に取得することが可能である。
% 求められてはないかな。。。。君がそういうのがあると有効と思ってるくらい。
% **ってなんなんだ?
% そもそも、OCR の話と銀行とかの連携って、別の話やで。。。
また、ATMや銀行との連携を補完するため、現金やクレジットカードを利用した取引も、
レシートがあればデータとして記録可能であることが本論文の特徴である。さらに、レシートや明細書がない取引(例:自動販売機での購入)に対応するため、
手動入力機能を併設し、ATM連携に頼らないデータ管理の選択肢を提供する。
% 銀行はレシートじゃなくて、明細書かな。。
しかし、単にレシートの情報をそのまま記録するだけでは、家計管理における柔軟性が不十分である。
例えば、コンビニでの買い物において、「食費」として分類したい商品と「生活必需品」として分類したい商品が混在する場合がある。
本論文では、この問題を解決するために、レシートから抽出した情報を行単位で整理する仕組みを導入する。
具体的には、レシート内の1つの商品や金額を1行として取り扱い、それぞれの行を任意のカテゴリに振り分けられるようにする。
この方法により、利用者は商品ごとに適切な分類を簡単に行うことができる。

さらに、現行の家計簿アプリには、OCRを利用してレシートを取り込む機能を備えているものの、
データの分類や編集の柔軟性に欠けるものが多い。例えば、「Zaim」では写真からデータを取得することは可能であるが、
品目ごとの分類は手動で行う必要がある。また、「マネーフォワード」では自動的に見やすいグラフや表を作成するものの、
数字の入力が必要であり、分類作業を効率化する機能は限定的である。

本論文では、OCR技術を用いてレシートから抽出した情報を行単位で整理し、自由にカテゴリ分けできる仕組みを開発することを目的とする。
さらに、利用者が設定した分類に基づき、比較したい項目を直感的に把握できるカスタマイズ可能なグラフ作成機能を提供する。
このシステムにより、ATMや銀行との連携に依存せず、かつ柔軟で効率的な家計管理を実現することを目指す。

本論文の構成は以下の通りである。まず \ref{cha:related} 章では、関連研究について述べる。さらに、
\ref{cha:exfig} 章では、図表の作成の仕方を述べる。
最後に、\ref{cha:conclusion} 章では、本論文のまとめと今後の課題について述べる。

\end{document}
