%! TEX root = ../main.tex
\documentclass[main]{subfiles}

\begin{document}
\chapter{開発するWebアプリケーションの動作}
\label{cha:motion}
本章では\ref{cha:overview}章で述べた、アプリケーションの機能の動作や使用した際の流れを記述する。

まず、ユーザーはレシート画像をアップロードすることで、OCR技術を用いて品目、金額、および日付を自動的に抽出する。
なので、レシート画像のアップロードを図を用いながら説明していく。

レシート画像のアップロードは、専用のインターフェースを通じて実行される。
ユーザーはデバイスから画像を選択し、アップロードボタンをクリックすることで、OCR処理が開始される\ref{fig:upload}。
システムでは、オープンソースのOCRエンジンであるTesseract OCRを採用している。
Tesseract OCRは、高い認識精度を持つ光学文字認識ライブラリであり、多言語対応が可能である。
本システムでは、日本語のレシートを対象としているため、日本語モデル(jpn)を用いてOCR処理を実行している。
OCR処理では、アップロードされたレシート画像から文字情報を解析し、品目名、金額、日付などの構造化された支出情報を抽出する。
実際にアップロードされたレシート画像から文字情報を解析した図が\ref{fig:}。



しかし、OCR技術の精度はレシートの印刷状態や画像の品質に依存するため、日付が正しく抽出されない場合がある。
このような場合には、日付を手動で入力できるインターフェースを提供している、その図が\ref{fig:input}。
この手動入力インターフェースは、OCRの抽出結果が不完全である場合の補完手段として機能し、
ユーザーが正確な支出記録を保持できるように設計されている。

\begin{figure}[tb]
    \begin{center}
        \fbox{
            \includegraphics[width=.8\linewidth]%
            {../figures/upload.png}
        }
        \caption{画像のアップロード方法}
        \label{fig:upload}
    \end{center}
\end{figure}

\begin{figure}[tb]
    \begin{center}
        \fbox{
            \includegraphics[width=.8\linewidth]%
            {../figures/input.png}
        }
        \caption{手動で入力場合の入力欄}
        \label{fig:input}
    \end{center}
\end{figure}

次に、抽出されたデータの登録や編集、分類について説明する。
本システムでは、OCR処理や手動入力により取得された支出データをアプリケーション内に登録し、管理を実現している。
取得された支出データをアプリケーション内に登録した後の画面が図\ref{fig:Registration}。

\begin{figure}[tb]
    \begin{center}
        \fbox{
            \includegraphics[width=.8\linewidth]%
            {../figures/Registration.png}
        }
        \caption{支出を表示した図}
        \label{fig:Registration}
    \end{center}
\end{figure}

登録されたデータは、品目名、金額、日付、カテゴリの4つの要素から構成される。
ユーザーはOCRによって取得されたデータをそのまま登録できるだけでなく、必要に応じて品目や金額、日付を編集することが可能である。
これにより、OCRの誤認識や手動入力時のミスを容易に修正することができる。

また、登録された支出データにはカテゴリを割り当てる機能が付加されている。
カテゴリは「食費」「交通費」「娯楽」など、ユーザーが自由に設定できる形式となっており、支出データを視覚的かつ論理的に分類することが可能である。
カテゴリは手動で入力できるようになっており、ユーザーのニーズに応じた分類が可能である。
実際に入力した図\ref{fig:classification}。

\begin{figure}[tb]
    \begin{center}
        \fbox{
            \includegraphics[width=.8\linewidth]%
            {../figures/classification.png}
        }
        \caption{カテゴリ欄に入力}
        \label{fig:classification}
    \end{center}
\end{figure}

さらに、登録された支出データは日付ごとに整理され、カレンダー機能を用いて特定の日付の支出を簡単に確認できる。
この設計により、支出の発生時期や用途を把握するためのデータ管理が容易になると同時に、日次、月次の集計や分析が可能となっている。

次に、本システムの合計金額表示機能について説明する。
この機能は、登録された支出データを基に、カテゴリごとや日付ごとの合計金額を計算し、それらをユーザーに視覚的に提示するものである。

具体的には、支出データが登録されると、自動的にカテゴリや日付ごとに分類され、各カテゴリまたは各日付に対応する合計金額が算出される。
これらの合計金額は、表形式で表示されるため、ユーザーは直感的に確認することができる。
特に、日付ごとの合計金額表示機能は、日々の支出を追跡し、予算管理をサポートする上で有用である。
また、カテゴリごとの合計金額表示は、どの支出カテゴリに最も多くの金額が費やされているかを一目で把握することが可能である。

この設計により、ユーザーは支出データを単に記録するだけでなく、
そのデータを分析し、消費パターンを把握するための具体的な手がかりを得ることができる。
本機能は、支出情報の視覚化とユーザーの意思決定を支援するための重要な役割を果たしている。

\begin{figure}[tb]
    \begin{center}
        \fbox{
            \includegraphics[width=.8\linewidth]%
            {../figures/category_total.png}
        }
        \caption{カテゴリごとの合計}
        \label{fig:category_total}
    \end{center}
\end{figure}

\begin{figure}[tb]
    \begin{center}
        \fbox{
            \includegraphics[width=.8\linewidth]%
            {../figures/date_total.png}
        }
        \caption{日付ごとの合計}
        \label{fig:date_total}
    \end{center}
\end{figure}

本システムのカレンダー機能は、日付ごとの支出データを簡単に記録及び管理できるインターフェースを提供する。
ユーザーはカレンダーで日付を選択するだけで、その日に記録された支出データを確認できる。
また、特定の日付の支出情報をカテゴリごとに分けたり、合計金額を表示したりすることで、その日の消費状況を把握しやすくしている。
さらに、過去の支出データにも簡単にアクセスできるため、長期的な消費傾向の分析にも役立てることができる。

ユーザーがカレンダーから特定の日付を選択すると、その日付がフォーマットされ、システム内で保存されている支出データと照合される。
そして、選択された日付に紐付けられた支出項目、金額、カテゴリ、およびその合計金額が画面に表示される。
例えば、2024年12月15日を選択した場合、図\ref{fig:12-15}のように表示される。
その後、2024年12月14日に選択しなおした場合図\ref{fig:12-14}のように変更される。
支出データは「日付」をキーとして保存されており、OCRによって自動的に紐付けられるが、
OCRで日付が取得できなかった場合には、ユーザーが手動で日付を入力できるインターフェースも提供している。
これにより、データの完全性を維持しながら、ユーザーの利便性を確保している。

\begin{figure}[tb]
    \begin{center}
        \fbox{
            \includegraphics[width=.8\linewidth]%
            {../figures/12-15.png}
        }
        \caption{2024年12月15日の支出}
        \label{fig:12-15}
    \end{center}
\end{figure}

\begin{figure}[tb]
    \begin{center}
        \fbox{
            \includegraphics[width=.8\linewidth]%
            {../figures/12-14.png}
        }
        \caption{2024年12月14日の支出}
        \label{fig:12-14}
    \end{center}
\end{figure}





\end{document}