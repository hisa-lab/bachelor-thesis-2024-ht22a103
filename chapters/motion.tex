%! TEX root = ../main.tex
\documentclass[main]{subfiles}

\begin{document}
\chapter{開発するWebアプリケーションの動作}
\label{cha:motion}
本章では\ref{cha:overview}章で述べた、アプリケーションの機能の動作や使用した際の流れを記述する。

\section{アプリケーション使用の際の流れ}

本アプリケーションを使用する際の全体的な流れについて説明する。
まず、ユーザーはレシート画像をアップロードすることで、OCR技術を利用して品目、金額、日付などの情報を自動的に抽出する。
アップロードされたレシート画像は、OCRを用いて解析され、支出情報が構造化された形式で取得される

次に、OCR処理で抽出された支出データがアプリケーション内に登録される。
ユーザーは必要に応じて、品目名や金額、日付の編集を行い、データを正確な形に整えることができる。
また、支出データにカテゴリを手動で入力することで、個別の分類が可能となる。
この分類機能により、支出情報を視覚的かつ論理的に整理することができる。

その後、登録された支出データは、カレンダー機能を通じて日付ごとに整理される。
ユーザーはカレンダーから任意の日付を選択し、その日の支出情報や合計金額を確認することが可能である。
また、日付単位の情報だけでなく、カテゴリごとの支出合計も確認できるため、消費パターンを多角的に分析することができる。

最後に、視覚化されたデータは、支出の合計金額をカテゴリ別や日付別に表示し、ユーザーに直感的な消費状況の把握を可能にする。
これにより、日常的な家計管理から長期的な消費傾向の分析まで幅広く活用できる仕組みとなっている。

\section{画像アップロード時の動作}

ユーザーはレシート画像をアップロードすることで、OCR技術を用いて品目、金額、および日付を自動的に抽出する。
レシート画像のアップロードは、専用のインターフェースを通じて実行される。
ユーザーはデバイスから画像を選択し、アップロードボタンをクリックすることで、OCR処理が開始される。

\subsection{OCRについて}

OCR(Optical Character Recognition、光学文字認識)は、画像内の文字情報を抽出する技術を指します。
この技術は、スキャンされた文書、写真、レシート、名刺などの画像形式で保存されたテキストをデジタルデータとして取り込むために使用されます。
本アプリケーションのシステムでは、オープンソースのOCRエンジンであるTesseract OCRを採用している。
Tesseract OCRは、高い認識精度を持つ光学文字認識ライブラリであり、多言語対応が可能である。
本システムでは、日本語のレシートを対象としているため、日本語モデル(jpn)を用いてOCR処理を実行している。
OCR処理では、アップロードされたレシート画像から文字情報を解析し、品目名、金額、日付などの構造化された支出情報を抽出する。

\subsection{OCRで写真をアップロードした際の動作}

実際にレシート写真をアップロードし解説する。
レシート写真receipt.jpeg 図\ref{fig:receipt}を画像を選択から選択する。
その後、アップロードボタンをクリックする図\ref{fig:upload}とOCR処理が開始される。
OCRで図\ref{fig:receipt}からテキストを取得すると図\ref{fig:OCRresult}のように取得される。
このテキストにはレシートに記載されている品目名、金額、日付などの情報が含まれており、
本システムはこれを基に支出データを構造化して処理する。
取得されたテキストは行単位で解析され、品目名と金額を抽出し、適切なフォーマットで記録される。
行単位とは、OCRで抽出されたテキストデータを「1行ごと」に区切り、その各行を解析の基本単位として処理する方法を指す。
具体的には、OCRで生成された生テキストには改行文字が含まれており、これを基にテキストを分割し、1行ずつ解析を行う。
このプロセスにより、ユーザーはレシートの情報を効率的にデジタル化し、家計簿に反映させることが可能となる。

しかし、OCR技術の精度はレシートの印刷状態や画像の品質に依存するため、文字が正しく抽出されない場合がある。
このような場合には、品目名、金額、日付を手動で入力できるインターフェースを提供している、その図が\ref{fig:input}。
この手動入力インターフェースは、OCRの抽出結果が不完全である場合の補完手段として機能し、
ユーザーが正確な支出記録を保持できるように設計されている。

\begin{figure}[tb]
    \begin{center}
        \fbox{
            \includegraphics[width=.8\linewidth]%
            {../figures/receipt.jpeg}
        }
        \caption{レシート写真の例 receipt.jpeg}
        \label{fig:receipt}
    \end{center}
\end{figure}

\begin{figure}[tb]
    \begin{center}
        \fbox{
            \includegraphics[width=.8\linewidth]%
            {../figures/upload.png}
        }
        \caption{画像のアップロード方法}
        \label{fig:upload}
    \end{center}
\end{figure}

\begin{figure}[tb]
    \begin{center}
        \fbox{
            \includegraphics[width=.8\linewidth]%
            {../figures/OCRresult.png}
        }
        \caption{OCR解析結果}
        \label{fig:OCRresult}
    \end{center}
\end{figure}

\begin{figure}[tb]
    \begin{center}
        \fbox{
            \includegraphics[width=.8\linewidth]%
            {../figures/input.png}
        }
        \caption{手動で入力場合の入力欄}
        \label{fig:input}
    \end{center}
\end{figure}

\section{支出データの登録、編集、および分類について}

次に、抽出されたデータの登録や編集、分類について説明する。
本システムでは、OCR処理や手動入力により取得された支出データをアプリケーション内に登録し、管理を実現している。
取得された支出データをアプリケーション内に登録した後の画面の例の図を下に示す\ref{fig:Registration}。

\begin{figure}[tb]
    \begin{center}
        \fbox{
            \includegraphics[width=.8\linewidth]%
            {../figures/Registration.png}
        }
        \caption{支出を表示した図}
        \label{fig:Registration}
    \end{center}
\end{figure}

登録されたデータは、品目名、金額、日付、カテゴリの4つの要素から構成される。
ユーザーはOCRによって取得されたデータをそのまま登録できるだけでなく、必要に応じて品目や金額、日付を編集することが可能である。
これにより、OCRの誤認識や手動入力時のミスを容易に修正することができる。

また、登録された支出データにはカテゴリを割り当てる機能が付加されている。
カテゴリは「食費」「交通費」「娯楽」など、ユーザーが自由に設定できる形式となっており、支出データを視覚的かつ論理的に分類することが可能である。
カテゴリは手動で入力できるようになっており、ユーザーのニーズに応じた分類が可能である。
実際に入力した図\ref{fig:classification}。

\begin{figure}[tb]
    \begin{center}
        \fbox{
            \includegraphics[width=.8\linewidth]%
            {../figures/classification.png}
        }
        \caption{カテゴリ欄に入力}
        \label{fig:classification}
    \end{center}
\end{figure}

さらに、登録された支出データは日付ごとに整理され、カレンダー機能を用いて特定の日付の支出を簡単に確認できる。
この設計により、支出の発生時期や用途を把握するためのデータ管理が容易になると同時に、日次、月次の集計や分析が可能となっている。

次に、本システムの合計金額表示機能について説明する。
この機能は、登録された支出データを基に、カテゴリごとや日付ごとの合計金額を計算し、それらをユーザーに視覚的に提示するものである。

具体的には、支出データが登録されると、自動的にカテゴリや日付ごとに分類され、各カテゴリまたは各日付に対応する合計金額が算出される。
これらの合計金額は、表形式で表示されるため、ユーザーは直感的に確認することができる。
特に、日付ごとの合計金額表示機能は、日々の支出を追跡し、予算管理をサポートする上で有用である。
また、カテゴリごとの合計金額表示は、どの支出カテゴリに最も多くの金額が費やされているかを一目で把握することが可能である。

この設計により、ユーザーは支出データを単に記録するだけでなく、
そのデータを分析し、消費パターンを把握するための具体的な手がかりを得ることができる。
本機能は、支出情報の視覚化とユーザーの意思決定を支援するための重要な役割を果たしている。

\begin{figure}[tb]
    \begin{center}
        \fbox{
            \includegraphics[width=.8\linewidth]%
            {../figures/category_total.png}
        }
        \caption{カテゴリごとの合計}
        \label{fig:category_total}
    \end{center}
\end{figure}

\begin{figure}[tb]
    \begin{center}
        \fbox{
            \includegraphics[width=.8\linewidth]%
            {../figures/date_total.png}
        }
        \caption{日付ごとの合計}
        \label{fig:date_total}
    \end{center}
\end{figure}

\section{カレンダー機能の動作と役割}

本システムのカレンダー機能は、日付ごとの支出データを簡単に記録及び管理できるインターフェースを提供する。
ユーザーはカレンダーで日付を選択するだけで、その日に記録された支出データを確認できる。
また、特定の日付の支出情報をカテゴリごとに分けたり、合計金額を表示したりすることで、その日の消費状況を把握しやすくしている。
さらに、過去の支出データにも簡単にアクセスできるため、長期的な消費傾向の分析にも役立てることができる。

ユーザーがカレンダーから特定の日付を選択すると、その日付がフォーマットされ、システム内で保存されている支出データと照合される。
そして、選択された日付に紐付けられた支出項目、金額、カテゴリ、およびその合計金額が画面に表示される。
例えば、2024年12月15日を選択した場合、図\ref{fig:12-15}のように表示される。
その後、2024年12月14日に選択しなおした場合図\ref{fig:12-14}のように表示されるデータが変更される。
支出データは「日付」をキーとして保存されており、OCRによって自動的に紐付けられるが、
OCRで日付が取得できない場合には、ユーザーが手動で日付を入力できるインターフェースも提供している。
これにより、データの完全性を維持しながら、ユーザーの利便性を確保している。

\begin{figure}[tb]
    \begin{center}
        \fbox{
            \includegraphics[width=.8\linewidth]%
            {../figures/12-15.png}
        }
        \caption{2024年12月15日の支出}
        \label{fig:12-15}
    \end{center}
\end{figure}

\begin{figure}[tb]
    \begin{center}
        \fbox{
            \includegraphics[width=.8\linewidth]%
            {../figures/12-14.png}
        }
        \caption{2024年12月14日の支出}
        \label{fig:12-14}
    \end{center}
\end{figure}





\end{document}