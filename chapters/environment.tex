%! TEX root = ../main.tex
\documentclass[main]{subfiles}

\begin{document}
\chapter{開発環境}
\label{cha:environment}
本章では、アプリケーションの開発に使用した環境について説明する。

開発言語には、主にJavaScriptおよびReactを採用した。
Reactは、コンポーネントベースの設計が可能であり、動的なUI構築に適していることから、
ユーザビリティを重視した本アプリケーションにおいて有効と判断した。
また、状態管理にはReactのuseStateおよびuseEffectといったフックを利用している。

OCR処理には、オープンソースのOCRエンジンであるTesseract.jsを利用した。
Tesseract.jsは、クライアントサイドで動作するJavaScriptベースのライブラリであり、
Webアプリケーションへの組み込みが容易である点が利点である。

データの保存には、Web Storage APIの一つであるlocalStorageを使用し、
ユーザーの入力情報およびOCR結果をブラウザに一時保存できるようにした。

さらに、本アプリケーションのソースコードはGitHub上でバージョン管理を行っている。
GitHubは、分散型バージョン管理システムGitを基盤とした開発プラットフォームであり、
コードの履歴管理、バグ追跡、ブランチによる機能追加の試行などが可能である。
GitHubを採用したことでコードの可搬性と再現性を確保するとともに、
今後の保守、改良にも対応しやすい環境を整備した。

下に開発環境の簡単な図\ref{fig:environment}を示す

\begin{figure}[tb]
    \begin{center}
        \fbox{
            \includegraphics[width=.8\linewidth]%
            {../figures/environment.png}
        }
        \caption{開発環境の簡単な図}
        \label{fig:environment}
    \end{center}

\end{figure}

\end{document}