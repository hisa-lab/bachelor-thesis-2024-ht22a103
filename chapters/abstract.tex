%! TEX root = ../main.tex
\documentclass[main]{subfiles}

\begin{document}

\begin{abstract}
本論文では、OCR(光学文字認識)技術を活用して、レシート情報をデジタル化し、
支出データを直感的に管理できる家計簿アプリケーションの開発を行った。        
既存の家計簿アプリケーションは、レシート情報の取り込みや詳細な支出管理において課題があり、
特に紙媒体のレシートをデジタル化して管理する機能が十分に整備されていない。

そこで、本論文ではOCR技術を利用し、レシートから品目名、金額、日付を自動抽出し、
支出データを整理する機能を実装した家計簿アプリケーションを開発した。
さらに、カレンダー形式で支出データを表示し、特定の日付の支出状況を簡単に把握できる機能や、
カテゴリ単位で支出を分類及び集計する機能を提供し、視認性と利便性の向上を図った。

また、本アプリケーションの画面を数名に閲覧してもらい、操作性や画面構成に関する意見を収集した。
これにより、UI設計における設計方針の妥当性や改善すべき点についての手がかりを得ることができた。
あわせて、今後の機能改良やユーザー体験の向上に向けた方針を検討する上で有益な知見が得られた。
        \keywords % 主な用語
        OCR\quad
        Web アプリケーション\quad
        家計簿アプリケーション\quad
\end{abstract}

\end{document}