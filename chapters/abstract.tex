%! TEX root = ../main.tex
\documentclass[main]{subfiles}

\begin{document}

\begin{abstract}

    既存の家計簿アプリケーションは、レシート情報の取り込みや詳細な支出管理において課題があり、
    特に紙媒体のレシートをデジタル化して管理する機能が十分に整備されていない。
    本論文では、OCR(光学文字認識)技術を活用して、レシート情報を効率的にデジタル化し、
    支出データを直感的に管理できる家計簿アプリケーションの開発を行った。
    本アプリケーションでは、レシートから品目名、金額、日付を自動抽出し、支出データを整理する機能を実装した。

    さらに、カレンダー形式で支出データを表示し、特定の日付の支出状況を簡単に把握できる機能や、
    カテゴリ単位で支出を分類及び集計する機能を提供し、視認性と利便性の向上を図った。
    アンケート評価では、支出管理の利便性やカテゴリ分け機能が好評を得た一方で、
    長期的な支出の集計機能や一括でのレシート読み取り機能の不足が課題として挙げられた。

        \keywords % 主な用語
        OCR\quad
        家計簿アプリケーション\quad
\end{abstract}

\end{document}