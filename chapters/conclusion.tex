%! TEX root = ../main.tex
\documentclass[main]{subfiles}

\begin{document}

\chapter{まとめと今後の課題}
\label{cha:conclusion}
本研究では、OCR(光学文字認識)技術を活用して、紙媒体のレシートから品目名・金額・日付といった支出情報を自動で抽出し、  
そのデータをカレンダー形式やカテゴリ別に表示・集計できるWebアプリケーションを開発した。
ユーザーはレシート画像をアップロードすることで、支出データを簡単に記録・整理できるようになり、  
さらにカテゴリごとの分類機能や、日付ごとの支出一覧・合計金額の表示機能により、視覚的かつ論理的に家計管理を行えるよう設計されている。
また、手動での入力・編集機能やカレンダーインターフェースなど、OCRが不完全な場合でも柔軟に運用可能な構造とし、  
実際の利用を想定した機能の実装を進めた。
今後の課題としては、対応できるレシートのフォーマット拡張や、OCRの精度向上が挙げられる。  
さらに、複数レシートの一括登録や、支出の長期的な集計機能、通知機能など、ユーザーの利便性を高める機能の追加も検討すべきである。
本アプリケーションは、日常の支出管理を支援するツールとして一定の有効性を持つと考えられ、  
さらなる改良を重ねることで、より幅広いユーザーに対応可能な家計簿システムとしての発展が期待される。

\end{document}