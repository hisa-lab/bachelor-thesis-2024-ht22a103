%! TEX root = ../main.tex
\documentclass[main]{subfiles}

\begin{document}

\chapter{まとめと今後の課題}
\label{cha:conclusion}
本研究では、OCR(光学文字認識)技術を活用し、紙媒体のレシートから品目名・金額・日付といった支出情報を自動で抽出するWebアプリケーションを開発した。
このアプリケーションは、抽出した支出データをカレンダー形式やカテゴリ別に表示・集計する機能を備えている。
ユーザーはレシート画像をアップロードすることで支出データを簡単に記録・整理でき、さらにカテゴリごとの分類機能や、
日付ごとの支出一覧および合計金額の表示機能によって、視覚的かつ論理的な家計管理を行えるよう設計されている。

また、本研究ではOCRの認識が不完全な場合でも手動入力・編集機能やカレンダーインターフェースを用いることで、柔軟に運用できる構造とした。
これにより、ユーザーは支出データの誤りを修正しながら正確に管理することが可能になる。
さらに、実際の利用を想定したこれらの機能の実装により、快適な使用環境を提供している点が本研究の特徴の一つである。

今後の課題としては、対応できるレシートのフォーマットを拡張し、OCRの認識精度をさらに向上させる必要がある。
加えて、ユーザーの利便性を高める機能の追加も重要である。
具体的には、複数レシートの一括登録機能、支出の長期的な集計・可視化機能、支出状況に応じた通知機能などが挙げられる。
これらの機能を実装することで、アプリケーションの実用性と応用範囲がさらに広がると考えられる。
本アプリケーションは、日常の支出管理を支援するツールとして一定の有効性を持っており、
今後の改良によってより多様なユーザーに対応できる家計簿システムへと発展することが期待される。
\end{document}