%! TEX root = ../main.tex
\documentclass[main]{subfiles}

\begin{document}

\chapter{まとめと今後の課題}
\label{cha:conclusion}
本研究では、OCR(光学文字認識)技術を活用し、紙媒体のレシートから品目名・金額・日付といった支出情報を自動で抽出するWebアプリケーションを開発した。  
このアプリケーションは、抽出された支出データをカレンダー形式やカテゴリ別に表示・集計する機能を備えている。  
ユーザーはレシート画像をアップロードするだけで、支出データを簡単に記録・整理できる。  
また、カテゴリごとの分類機能や日付ごとの支出一覧・合計金額表示機能を通じて、視覚的かつ論理的な家計管理を実現している。  
これらの機能により、日常の金銭管理を効率化するツールとしての有効性が示された。

さらに、本研究ではOCRの認識が不完全な場合やレシートや明細書が発行されない場合に備え、手動入力および編集機能を提供する設計とした。  
ユーザーはOCRによって誤認識された情報を修正することで、正確な支出記録を保持できる。  
加えて、カレンダーインターフェースにより、日付ごとのデータ閲覧や再編集が直感的に行えるように工夫した。  
これらの補完機能は、OCR精度の限界を補いながら、実用的な運用を可能にしている。  
実際の利用場面を想定したインターフェース設計と柔軟な操作性は、本研究の特徴の一つである。

今後の課題としては、OCRが対応可能なレシートフォーマットの拡張が挙げられる。  
現状では、特定の3店舗のレシートに限定して解析を行っているが、他店舗や異なるフォーマットへの対応が求められる。  
また、OCRの認識精度そのものの向上も重要であり、より高度な前処理や学習モデルの適用が必要とされる。  
これに加えて、ユーザーの利便性を高めるための新機能の追加も今後の改善点として重要である。

具体的には、複数レシートの一括登録機能や、支出の長期的な集計・可視化を行うダッシュボードの実装が挙げられる。  
さらに、ユーザーの支出傾向に応じて通知を行うリマインダー機能や、他サービスとの連携を視野に入れたデータエクスポート機能なども有効である。  
これらの機能を追加することで、アプリケーションの実用性と応用範囲はさらに広がると考えられる。

本アプリケーションは、日常的な支出管理を支援するWebツールとして一定の有用性を示しており、今後の改良によってさらなる発展が期待される。  
多様なレシート形式への対応や分析機能の強化、ユーザーインターフェースの改善を通じて、より多くの利用者にとって実用的な家計簿システムへと成長させることが、本研究の今後の展望である。

\end{document}