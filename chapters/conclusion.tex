%! TEX root = ../main.tex
\documentclass[main]{subfiles}

\begin{document}

\chapter{まとめと今後の課題}
\label{cha:conclusion}

本本研究では、OCR技術を活用した家計簿アプリケーションを開発し、その動作確認とユーザー評価を通じて、アプリケーションの有用性と課題を明らかにした。
本システムは、レシート画像から品目、金額、日付を自動抽出するOCR機能を中心に、手動入力、データ編集、
カテゴリごとの集計表示、カレンダーによる日付管理など、多岐にわたる機能を備えることで、
紙媒体の支出情報を効率的にデジタル化し、ユーザーが直感的かつ容易に支出データを管理できる仕組みを提供した。

一方で、本システムには対応可能なレシート形式が限定的であるという課題がある。
特に、レイアウトが複雑なレシートや手書きが含まれるレシートに対しては、OCRの精度が低下し、適切にデータを抽出できない場合がある。
この問題を解決するためには、OCRモデルの改善や追加機能の導入が求められる。

今後の課題としては、ユーザーから寄せられた要望を基にした新機能の実装や既存機能の改良に加え、
対応可能なレシート形式を拡大することが挙げられる。また、さらなるユーザー調査やフィールドテストを実施し、
システムの実用性と完成度を高めることで、幅広いユーザーにとって利便性の高いアプリケーションを目指す必要がある。

\end{document}