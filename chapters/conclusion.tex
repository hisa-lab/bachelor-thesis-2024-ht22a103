%! TEX root = ../main.tex
\documentclass[main]{subfiles}

\begin{document}

\chapter{まとめと今後の課題}
\label{cha:conclusion}

本研究では、OCR技術を活用した家計簿アプリケーションを開発し、その動作やユーザー評価を通じて、アプリケーションの有用性と課題を検討した。
本システムは、レシート画像から品目、金額、日付を自動抽出するOCR機能を中心に、手動入力やデータ編集機能、
カレンダーによる日付管理、カテゴリごとの集計表示など、多岐にわたる機能を提供することで、
紙媒体の支出情報を効率的にデジタル化し、ユーザーが容易に管理できる仕組みを実現した。

ユーザーアンケートの結果、直感的な操作性や使いやすさに対して良好な評価が得られた一方で、さらなる改善点も浮き彫りとなった。
具体的には、1週間、1ヶ月、年間といった長期的な集計機能、一括でのレシート処理機能、メモ機能、支出通知機能などが求められている。
また、画面デザインの改善や情報配置の最適化も、システムの利用性を向上させるための重要な課題として挙げられた。

今後の課題として、ユーザー要望に基づいた新機能の実装と既存機能の改良に加え、
さらなるユーザー調査やフィールドテストを通じてシステムの完成度を高める必要がある。

\end{document}